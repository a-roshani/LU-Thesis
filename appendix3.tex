% !TeX root=main.tex

\chapter{وارد کردن کدهای برنامه‌نویسی}\label{App:Programming Code}
\thispagestyle{empty}

در این بخش نمونه مثال‌هایی از ورود کدهای برنامه نویسی ارائه خواهد شد. برای این منظور می‌توان از دو محیط زیر استفاده کرد. محیط اول مربوط به بسته 
\verb|listings|
 است که در آن تنظیمات مربوط به زبان برنامه‌نویسی به عنوان یک قابلیت اضافه وجود دارد. محیط دوم مربوط به بسته 
 \verb|verbatim|
 است. برای کسب اطلاعات بیشتر در مورد این دو بسته راهنمای آن‌ها را ببینید.
 
 محیط اول: زبان برنامه نویسی 
 \verb|Matlab|
 % به جای کلمه Matlab می توان متناسب با زبان مورد نظر از R , Python, SAS, PHP, C++, SQL, Java و ... استفاده کرد
\begin{latin}
\begin{lstlisting}[language=Matlab]
n = normrnd([1 2 3;4 5 6],0.1,2,3)
\end{lstlisting}
\end{latin}

محیط دوم: زبان برنامه 
 \verb|R|
 
 \begin{latin}
 \begin{verbatim}
set.seed(99)
x <- rnorm(100)
plot(density(x))
\end{verbatim}
 \end{latin}
 